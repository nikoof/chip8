\documentclass[a4paper]{article}

\usepackage{fontspec}
\usepackage[romanian]{babel}
\usepackage{amsmath}
\usepackage{csquotes}

\usepackage[backend=biber]{biblatex}
\addbibresource{main.bib}

\usepackage{geometry}
\geometry{
  top=4cm,
  bottom=4cm,
  left=3.5cm,
  right=3.5cm,
}

\begin{document}
\begin{titlepage}
	\begin{center}
		\Large Proiect pentru obținerea atestării profesionale în informatică
		\vfill
		\LARGE\textbf{Octarou: Interpretor pentru limbajul de programare CHIP-8}

		\vspace{8pt}
		\Large Nicolas-Ștefan Bratoveanu \\
		\large Prof. coordonator Mihaela Stan

		\vfill
		\Large
		Colegiul Național „Vasile Alecsandri” Galați \\
		2023-2024
	\end{center}
\end{titlepage}

\tableofcontents
\newpage

\section{Introducere}
\subsection{Contextualizare}
\textit{Octarou} este un interpretor pentru limbajul de programare CHIP-8. Acesta este un limbaj de programare interpretat, dezvoltat
de către Joseph Weisbecker în 1977 pentru sisteme bazate pe microprocesorul RCA 1802. A fost inițial utilizată pe computerele COSMAC VIP și Telmac
1800. Limbajul a fost creat cu scopul de a permite dezvoltarea mult mai ușoară a jocurilor video pe aceste platforme, permițând utilizarea unor
instrucțiuni hexazecimale, în locul instrucțiunilor native\cite{langhoff}. Spre sfârșitul anilor '70, în spatele acestui limbaj se formase o comunitate activă
de dezvoltatori și utilizatori, care a luat naștere odată cu buletinul informativ \textit{VIPer} al revistei \textit{ARESCO}, ale cărei prime
ediții au dezvăluit codul pentru interpretorul original de CHIP-8.

CHIP-8 s-a răspândit pe alte platforme, precum computerele australiene DREAM 6800, ETI-660 și MicroBee, computerul finlandez menționat mai devreme,
Telmac 1800 și computerul canadian ACE VDU\cite{langhoff}.

Ulterior, au apărut interpretoare derivate și extensii la limbajul original, folosite, spre exemplu, pe calculatoare grafice (CHIP-48, SUPER-CHIP
pentru calculatoarele HP-48\cite{langhoff}). Aceste dispozitive aparțineau perioadei de după anii '80 și aveau, de regulă, mult mai multă putere de procesare
decât microcomputerele din deceniul precedent, cum ar fi COSMAC VIP-ul.

\subsection{Motivația temei}
La nivel de suprafață, alegerea acestei teme poate părea (cel puțin) dubioasă. Cu toate că este, într-adevăr, o temă destul de obscură, consider
că este extrem de valoroasă. Implementarea unui astfel de sistem este adesea recomandată ca prim pas în lumea dezvoltării de emulatoare\cite{langhoff}.

\section{Mediul de dezvoltare}
\subsection{Nix}
\subsection{Rust}
\subsection{Librăriile \texttt{eframe} și \texttt{egui}}

\section{Structura aplicației}
\subsection{Limbajul \texttt{CHIP-8}}
\subsection{Interpretarea \texttt{CHIP-8}}
\subsection{Extensia \texttt{SUPERCHIP}}
\subsection{Interfața grafică}

\section{Îmbunătățiri}

\section{Concluzii}

\newpage
\printbibliography[title=\section{Bibliografie}]

\end{document}
